\begin{abstract}

Recentemente il mercato dello shopping online sta acquisendo sempre più rilevanza, superando i limiti della compravendita in negozi fisici e in alcuni ambiti rimpiazzandola. Con la crescita degli acquisti, cresce anche la mole di dati che venditori, produttori, pubblicitari e gestori di piattaforme di e-Commerce si trovano a dover processare per ottenere informazioni sulla natura delle transazioni, dei clienti che le producono e sui trend mercato.

Una parte fondamentale di questi dati è costituita da quelli prodotti dai consumatori stessi dopo aver effettuato l'acquisto: opinioni, recensioni e valutazioni sul prodotto ed in generale sull'esperienza di acquisto.

Sentiment Analysis è una materia che sfrutta dati di questa natura (denominati VOC: \textit{Voice of the Costumer}) per estrarre, quantificare e studiare informazioni soggettive in modo sistematico.

Tra i campi che beneficiano di questi strumenti troviamo: sviluppo di strategie di marketing, sistemi di raccomandazioni, \textit{brand monitoring}, servizio clienti e ricerche di mercato.

In questo lavoro, analizziamo un insieme di recensioni pubblicate su Amazon su articoli della categoria "Cellulari e accessori correlati" per uno studio esplorativo, estraendo dati statistici ed evoluzioni temporali sulla natura delle recensioni e delle valutazioni numeriche annesse. Procediamo poi nell'addestrare modelli di Machine Learning (Logistic Regression e Naive Bayes) per valutare la loro efficacia nell'identificare correttamente il sentimento generale delle recensioni, analizzandone poi le metriche e cercando di individuare i migliori iperparametri.
Infine per i sei prodotti più recensiti applichiamo una tecnica di Topic Analysis per individuare i cluster di argomenti.


\end{abstract}